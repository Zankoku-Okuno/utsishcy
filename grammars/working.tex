% !TEX TS-program = xelatex
% !TEX encoding = UTF-8

\documentclass[12pt]{book} % use larger type; default would be 10pt

\usepackage{fontspec} % Font selection for XeLaTeX; see fontspec.pdf for documentation
%\defaultfontfeatures{Mapping=tex-text} % to support TeX conventions like ``---''
%\usepackage{xunicode} % Unicode support for LaTeX character names (accents, European chars, etc)
%\usepackage{xltxtra} % Extra customizations for XeLaTeX

\setmainfont{CJunicode} % set the main body font (\textrm), assumes Charis SIL is installed
%\setsansfont{Deja Vu Sans}
%\setmonofont{Deja Vu Mono}

% other LaTeX packages.....
\usepackage{geometry}
\geometry{letterpaper} % or a4paper (US) or a5paper or....
%\usepackage[parfill]{parskip} % Activate to begin paragraphs with an empty line rather than an indent

\usepackage{graphicx} % support the \includegraphics command and options


\usepackage{float}
\usepackage{subcaption}





\title{Utsišcy (utsiʃkɛ) \\ {\Large Draft Grammar}}
\author{Okuno Zankoku}
%\date{} % Activate to display a given date or no date (if empty),
         % otherwise the current date is printed 

\begin{document}
\maketitle

\chapter{Demography and Ethnography}
So, what is this language, Utsišcy?

It's certainly an art language, as its only purpose is my enjoyment and hopefully someone else's as well.
If I describe it as a naturalistic language, that's because I think the assymmetries of natural languages are aesthetically pleasing in a wabi-sabi sort of way.
I've spared little thought for relative infrequencies of the various features I've chosen to include in preference to what I find fun, as long as the result is ``harmonious''.

Many naturalistic artlangs are also embedded in a conworld and come with a conculture, but that's not what I'm really interested in.
This language has no conceit (as fun as they can be); it is made up by a single person, and I present it that way.
I sometimes mention Utsišcy's ``history'' (which it doesn't have), it is only to allow me to explore the relation of some twist to the rest of the language and understand how to balance it.

At the end of the day, I'm not even sure I'm conlanging: I think I'm con-ideolecting.





\chapter{Sound System}

\section{Phonemics}

The number of phonemes available in Utsišcy varies depending on your theoretical predelictions.
There are several clusters analyzable as affricates and most consonants may be labialized or palatalized, which quite explodes the count of consonants.
The perspective I take is that apparently affricate, labialized, and palatalized consonants are underlyingly clusters.
This gives 35 contoid and 6 vocoid phonemes, a large inventory even at this conservative analysis.

\begin{table}[H]
\centering
\begin{subtable}[t]{.5\linewidth}
	\centering
	\begin{tabular}{ccccc}
		\hskip 0.7em m & \hskip 0.8em n \\
		p b & t d & c ɟ & k \\
		\hskip 0.8em ɓ & t' ɗ & c' ʄ & k' \\
		f v & s z & ʃ ʒ & x \\
		& \hskip 0.8em s' & \hskip 0.8em ʃ\,' & x' \\
		& r̥ r & & & ʀ \\
		& l̥ l & ʎ̥ ʎ \\
		W && j̊ j \\
	\end{tabular}
	\caption{contoids}\label{t:phoneme-contoids}
\end{subtable}
\hskip -6em
\begin{subtable}[t]{.5\linewidth}
	\centering
	\begin{tabular}{ccc}
		i & ɨ u \\
		ɛ & a ɔ \\
	\end{tabular}
	\caption{vocoids}\label{t:phoneme-vocoids}
\end{subtable}
\caption{Phoneme Intentory}\label{t:phonemes}
\end{table}

Several consistently distinctive features support the inventory size.
First, stops have four voicings in most places: unvoiced, voiced, ejective, and implosive; the only exceptions are the cross-linguistically common gaps /p', g, ɠ/.
Likewise, fricatives have three voicings in most places: unvoiced, voiced, and ejective.
The presence of ejective fricatives is unusual, but the gaps in fricatives /f', ɣ/ are also common.

Second, there is a distinction between palatals and velars.
Historically, this was a distinction between velar and uvular contoids which underwent fronting.
This brings some sanity to the appearance of a /ʄ/ < */k'/ phoneme without a corresponding */k'/ < **/q'/.

Third, Utsišcy consistently distinguishes between voiced and unvoiced variants of liquids /r̥, l̥, ʎ̥, j̊/ vs. /r, l, ʎ, j/.
The only exception is /W/ which is underspecified w.r.t. voicing.
In fact, /W/ might be analyzed as a ``bound phoneme'', i.e. it only appears in order to labialize another consonant (TODO: maybe to diphthongize a vowel, I haven't decided if I want to concede anything to diphthongs).
In this capacity, it takes on the voice of whatever contoid to which it is adjacent.

A careful reader may note the presence of three trills, but it should be noted that only /r̥, r/ pattern as rhotics; /ʀ/ < */ʁ/, and so patterns as a voiced fricative (TODO and probably also has [ʁ] as an allophone).

In comparison to the contoids, the vocoid system is reserved.
It is similar to the Russian vocoid system: a simple /i, e, a, o, u/ with a center-high vowel /ɨ/.

\subsection{TODO Contrastive Hierarchy}

As suggested by chart \ref{t:phoneme-vocoids}, the constrastive hierarchy for vocoids distinguishes high/low, front/back, and in the back vocoids unround/round.
Clearly, [+round] is marked, and [high/low], [front/back] > [round].
However, it is not clear which of high/low is marked, or which of front/back is marked, or which of those two distinctions are primary.

I'm just collecting some things I've noticed about contoids:
\begin{itemize}
\item Before trying to create a hierarchy, I should probably note the features that each phoneme even has first.
\item probably [labial/lingual] > [coronal/dorsal] > [velar/palatal]
\item  I think I must have [labial] > voicing, since /W/ doesn't have voicing; likely also [obstruent] > voicing as well
\end{itemize}





\section{Orthography}

Utsišcy's orthographies are mostly one-to-one phonemic, therefore I've decided to explain them here.
Utsišcy has both a romanization and a cyrillization.
Further, there is an ``asciization''---using only the glyphs of the ASCII character set---for use when diacritics are cumbersome or impossible to input to a computer system.

The non-phonemic exceptions in orthography capture regular phonetic processes in a way that is consistent with the script in use.
The phonemes /l̥, ʎ̥/, normally spelled «lh, llh» drop the <<-h>> when the phonetic context would imply they mjust be unvoiced.

There are a few special aesthetic considerations also.
In particular, /tʃ, cʃ/ clusters normally present as [tʃ], and so share the spelling «cš».
Additionally, the cyrillization leverages existing characters «ц, ч» for the clusters /ts/, [tʃ] since doing otherwise would read as highly unusual to anyone already familiar with cyrillic script.
The romanization does not use «q» or «qu» for /kʷ/ however.


TODO: if /j/ is going to attach to contoids, say as in /njɛt/, I'm not sure I want it spelled «nget», though that might be quirky/shibbolethy enough to be fun.
Further, using «w» just seems heavyweight to me. How could I get away writing /ʃʷatʃ/ not as «swacš», but as «suacš», «svacš», or something without compromising consonant clusters or not-diphthongs?
«ng̊et» and «sv̊acš»?

TODO when do I use capitalization?

\begin{table}[H]
\centering
\begin{subtable}[t]{.5\linewidth}
	\centering
	\begin{tabular}{ccccc}
		\hskip 0.7em m & \hskip 0.8em n \\
		p b & t d & c j & k \\
		\hskip 0.8em b' & t' d' & c' j' & k' \\
		f v & s z & š ž & x \\
		& \hskip 0.8em s' & \hskip 0.8em š' & x' \\
		& ř r & & & x̌ \\
		& lh l & ļ ļh \\
		w && ç g \\
	\end{tabular}
	\caption{consonants}\label{t:roman-consonants}
\end{subtable}
\hskip -6em
\begin{subtable}[t]{.5\linewidth}
	\centering
	\begin{tabular}{ccc}
		i & y u \\
		e & a o \\
	\end{tabular}
	\caption{vowels}\label{t:roman-vowels}
\end{subtable}

\begin{subtable}[t]{.5\linewidth}
	\centering
	\begin{tabular}{rc @{\hskip 3em} rc}
		tʃ & cš					&	tʃ\,' & cš' \\
		dʒ & ǰ \\
		C[-voice]l̥ & Cl̥		&	C[-voice]ʎ̥ & Cll \\
	\end{tabular}
	\caption{clusters (TODO)}\label{t:roman-clusters}
\end{subtable}
\caption{Romanization}\label{t:romanization}
\end{table}


\begin{table}[H]
\centering
\begin{subtable}[t]{.5\linewidth}
	\centering
	\begin{tabular}{ccccc}
		\hskip 0.7em m & \hskip 0.8em n \\
		p b & t d & c j & k \\
		\hskip 0.8em b' & t' d' & c' j' & k' \\
		f v & s z & sh zh & x \\
		& \hskip 0.8em s' & \hskip 1.2em sh' & x' \\
		& rh r & & & xh \\
		& lh l & llh ll \\
		w && ch g \\
	\end{tabular}
	\caption{consonants}\label{t:ascii-consonants}
\end{subtable}
\hskip -6em
\begin{subtable}[t]{.5\linewidth}
	\centering
	\begin{tabular}{ccc}
		i & y u \\
		e & a o \\
	\end{tabular}
	\caption{vowels}\label{t:ascii-vowels}
\end{subtable}

TODO notes on the odd character choices (j, g, ç, x̌, ř).
TODO comment on cedilla vs. comma below in unicode.


\begin{subtable}[t]{.5\linewidth}
	\centering
	\begin{tabular}{rc @{\hskip 3em} rc}
		tʃ & csh		&	tʃ\,' & csh' \\
	\end{tabular}
	\caption{clusters (TODO)}\label{t:ascii-clusters}
\end{subtable}
\caption{Asciization}\label{t:asciization}
\end{table}

\begin{table}[H]
\centering
\begin{subtable}[t]{.5\linewidth}
	\centering
	\begin{tabular}{ccccc}
		\hskip 0.7em м & \hskip 0.8em н \\
		п б & т д & ть дь & к \\
		\hskip 0.8em бӏ & тӏ дӏ & тьӏ дьӏ & кӏ \\
		ф в & с з & ш ж & х \\
		& сӏ \hskip 0.8em & шӏ \hskip 0.8em & хӏ \\
		& рхӏ р & & & гӏ \\
		& лхӏ л & лхӏь ль \\
		оъ && йхӏ й \\
	\end{tabular}
	\caption{consonants}\label{t:cyrillic-consonants}
\end{subtable}
\hskip -6em
\begin{subtable}[t]{.5\linewidth}
	\centering
	\begin{tabular}{ccc}
		и & і у \\
		е & а о \\
	\end{tabular}
	\caption{vowels}\label{t:cyrillic-vowels}
\end{subtable}

\begin{subtable}[t]{.5\linewidth}
	\centering
	\begin{tabular}{rc @{\hskip 3em} rc}
		ts & ц		&	ts' & цъ \\
		tʃ & ч		&	tʃ\,' & чъ \\
	\end{tabular}
	\caption{clusters (TODO)}\label{t:cyrillic-clusters}
\end{subtable}
\caption{Cyrillization (FIXME: tentative)}\label{t:cyrillization}
\end{table}

In the Cyrillization, probably replace yer (ъ) with palochka (ӏ), dependent on fonts (it's look a lot better than дьъ...).
I'm not sure about using «г» for /ʀ/.
Perhaps љ in place of ль.
I want ҋ in place of йхӏ: it's so much easier to read.
I know some language uses рхӏ exactly as I am here, but transporting that convension to the other liquids, though consistent, makes for some ugly n-graphs.
It's a bit extreme to use a digraph оъ for a labialization phoneme.







\end{document}
