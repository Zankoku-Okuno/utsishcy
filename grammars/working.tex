% !TEX TS-program = xelatex
% !TEX encoding = UTF-8

\documentclass[12pt]{book} % use larger type; default would be 10pt

\usepackage{fontspec} % Font selection for XeLaTeX; see fontspec.pdf for documentation
%\defaultfontfeatures{Mapping=tex-text} % to support TeX conventions like ``---''
%\usepackage{xunicode} % Unicode support for LaTeX character names (accents, European chars, etc)
%\usepackage{xltxtra} % Extra customizations for XeLaTeX

\setmainfont[Numbers=OldStyle]{CJunicode} % set the main body font (\textrm), assumes Charis SIL is installed
%\setsansfont{Deja Vu Sans}
%\setmonofont{Deja Vu Mono}

\usepackage{scalefnt}
\newcommand{\mathipa}[1]{\textrm{\scalefont{1.08}\selectfont #1}} % scaled to 1.1 also seems ok


% other LaTeX packages.....
\usepackage{geometry}
\geometry{letterpaper} % or a4paper (US) or a5paper or....
%\usepackage[parfill]{parskip} % Activate to begin paragraphs with an empty line rather than an indent

\usepackage{graphicx} % support the \includegraphics command and options


\usepackage{float}
\usepackage{subcaption}

\usepackage{qtree}
\usepackage{mathtools}



\title{Utsišcy /ˈutsiʃkɨ/ \\ {\Large Draft Grammar}}
\author{Okuno Zankoku}
%\date{} % Activate to display a given date or no date (if empty),
         % otherwise the current date is printed 

\begin{document}
\maketitle





\chapter{Demography and Ethnography}
So, what is this language, Utsišcy?

It's certainly an art language, as its only purpose is my enjoyment and hopefully someone else's as well.
If I describe it as a naturalistic language, that's because I think the assymmetries of natural languages are aesthetically pleasing in a wabi-sabi sort of way.
I've spared little thought for relative infrequencies of the various features I've chosen to include in preference to what I find fun, as long as the result is ``harmonious''.

Many naturalistic artlangs are also embedded in a conworld and come with a conculture, but that's not what I'm really interested in.
This language has no conceit (as fun as they can be); it is made up by a single person, and I present it that way.
I sometimes mention Utsišcy's ``history'' (which it doesn't have), it is only to allow me to explore the relation of some twist to the rest of the language and understand how to balance it.

At the end of the day, I'm not even sure I'm conlanging: I think I'm con-ideolecting.


FIXME: tʃ' needs to look more like tʃ\,'


\chapter{Sound System}

\section{Phonemics}

The number of phonemes available in Utsišcy varies depending on your theoretical predelictions.
There are several clusters analyzable as affricates and most consonants may be labialized or palatalized, which quite explodes the count of consonants.
The perspective I take is that apparently affricate, labialized, and palatalized consonants are underlyingly clusters.
This gives 35 contoid and 6 vocoid phonemes, a large inventory even at this conservative analysis.

\begin{figure}[H]
\centering
\begin{subtable}[t]{0.5\linewidth}
	\centering
	\begin{tabular}{ccccc}
		\hskip 0.7em m & \hskip 0.8em n \\
		p b & t d & c ɟ & k \\
		\hskip 0.8em ɓ & t' ɗ & c' ʄ & k' \\
		f v & s z & ʃ ʒ & x \\
		& \hskip 0.8em s' & \hskip 0.8em ʃ' & x' \\
		& r̥ r & & & ʀ \\
		& l̥ l & ʎ̥ ʎ \\
		W && j̊ j \\
	\end{tabular}
	\caption{contoids}\label{t:phoneme-contoids}
\end{subtable}
\hskip -6em
\begin{subtable}[t]{0.5\linewidth}
	\centering
	\begin{tabular}{ccc}
		i & ɨ u \\
		ɛ & a ɔ \\
	\end{tabular}
	\caption{vocoids}\label{t:phoneme-vocoids}
\end{subtable}
\caption{Phoneme Intentory}\label{t:phonemes}
\end{figure}

Several consistently distinctive features support the inventory size.
First, stops have four voicings in most places: unvoiced, voiced, ejective, and implosive; the only exceptions are the cross-linguistically common gaps /p', g, ɠ/.
Likewise, fricatives have three voicings in most places: unvoiced, voiced, and ejective.
The presence of ejective fricatives is unusual, but again the gaps /f', ɣ/ are common.

Second, there is a distinction between palatals and velars.
Historically, this was a distinction between velar and uvular contoids which underwent fronting.
This brings some sanity to the appearance of a /ʄ/ < */k'/ phoneme without a corresponding */k'/ < **/q'/.

Third, Utsišcy consistently distinguishes between voiced and unvoiced variants of liquids /r̥, l̥, ʎ̥, j̊/ vs. /r, l, ʎ, j/.
The only exception is /W/ which is underspecified w.r.t. voicing.
In fact, /W/ might be analyzed as a ``bound phoneme'', i.e. it only appears in order to labialize another consonant (TODO: maybe to diphthongize a vowel, I haven't decided if I want to concede anything to diphthongs).
In this capacity, it takes on the voice of whatever contoid to which it is adjacent.

A careful reader may note the presence of three trills, but it should be noted that only /r̥, r/ pattern as rhotics; /ʀ/ < */ʁ/, and so patterns as a voiced fricative (TODO and probably also has [ʁ] as an allophone).

In comparison to the contoids, the vocoid system is reserved.
It is similar to the Russian vocoid system: a simple /i, e, a, o, u/ with a center-high vowel /ɨ/.

\subsection{TODO Featural Analysis}

TODO list and describe distinctive features

\subsubsection{Contrastive Hierarchy}

Given that Utsišcy have very many vocoids, I've split the hierarchy into figs.\ \ref{fig:contrast-hier-top}--\ref{fig:contrast-hier-continuants}.
The hierarchy begins in fig.\ \ref{fig:contrast-hier-top}, which then will direct you to the figures for subhierarchies.
Marked values are consistently given with as ``+'' and branch to the left.

\begin{figure}[H]
\centering
\Tree [
	[.+contoid
		\qroof{\small Fig.\ \ref{fig:contrast-hier-sonorants}}.+sonorant
		[.-sonorant
			\qroof{\small Fig.\ \ref{fig:contrast-hier-continuants}}.+continuant
			\qroof{\small Fig.\ \ref{fig:contrast-hier-stops}}.-continuant
		]
	]
	[.-contoid
		[.+obstruent
			[.{\footnotesize +labial} ◌ʷ ]
			[.{\footnotesize -labial} ◌ʲ ]
		]
		\qroof{\small Fig.\ \ref{fig:contrast-hier-vowels}}.-obstruent
	]
]
\caption{Contrastive Hierarchy}\label{fig:contrast-hier-top}
\end{figure}

Even a cursory inspection of the top level of the hierarchy shows that the inventory is light on vocoids and heavy on non-sonorant contoids.
Note that the non-sonorant vocoids /◌ʷ, ◌ʲ/ appear here; as they are a small category, their ink on the page is easily overlooked.
Although the contraints of the tree have forced me to imply one of \{contoid,~vocoid\} is marked, it's not entirely clear which should be.
Since the [sonorant] is natural for vocoids, but marked for contoids, we've used both [sonorant] and [obstruent] features, but these are simple antonyms of each other.

\begin{figure}[H]
\centering
\Tree [.{-contoid, +sonorant}
	[.{\footnotesize +high}
		[.{\footnotesize +back}
			[.{\scriptsize +round} u ]
			[.{\scriptsize -round} ɨ ]
		]
		[.{\footnotesize -back} i ]
	]
	[.{\footnotesize -high}
		[.{\footnotesize +back}
			[.{\scriptsize +round} ɔ ]
			[.{\scriptsize -round} a ]
		]
		[.{\footnotesize -back} ɛ ]
	]
]
\caption{Contrasive Hierarchy: vowels}\label{fig:contrast-hier-vowels}
\end{figure}

As suggested by fig. \ref{t:phoneme-vocoids}, vowels are organized into a 2$\times$2 space with an additional rounding contrast only in the back vowels.
Although it is not clear from phnological processes, tongue posture sugggests [high, back] are marked.\footnote{FIXME: Is this actually a thing? My guess is that there hasn't been enough research into the intersection of contrastive hierarchy and mouth posture to say; annoyingly, I don't think there's much research into mouth posture at all, except to inform accent training.}
Though I've had to decide on an order to draw the tree, given the symmetry between the two features it is likely that [high] $\sim$ [back] rather than one taking a higher spot than the other.

\begin{figure}[H]
\centering
\Tree [.{+contoid, -sonorant, -continuant} 
	[.{\footnotesize +labial}
		[.{\scriptsize +nonplosive} ɓ ] !\qsetw{1cm}
		[
			[.{\scriptsize +gc} ◌ ] !\qsetw{0.5cm}
			[
				[.{\scriptsize +voice} b ] !\qsetw{0.7cm}
				[ p ] !\qsetw{0.5cm}
			]
		]
	]
	[
		[.{\footnotesize +dorsal}
			[.{\footnotesize +velar} 
				[.{\scriptsize +nonplosive} ◌ ] !\qsetw{1.1cm}
				[
					[.{\scriptsize +gc} k' ] !\qsetw{0.5cm}
					[ k ] !\qsetw{0.5cm}
				]
			]
			[.{\footnotesize -velar}
				[.{\scriptsize +nonplosive} ʄ ] !\qsetw{1cm}
				[
					[.{\scriptsize  +gc} c' ] !\qsetw{0.5cm}
					[
						[.{\scriptsize +voice} ɟ ] !\qsetw{0.7cm}
						[ c ] !\qsetw{0.5cm}
					]
				]
			]
		]
		[.{\footnotesize -dorsal}
			[.{\footnotesize +nonplosive} ɗ ] !\qsetw{1.5cm}
			[.{\footnotesize -nonplosive}
				[.{\footnotesize +gc} t' ]
				[.{\footnotesize -gc}
					[.{\footnotesize +voice} d ]
					[.{\footnotesize -voice} t ]
				]
			]
		]
	]
]
\caption{Contrastive Hierarchy: stops}\label{fig:contrast-hier-stops}
\end{figure}

In the stops (fig.\ \ref{fig:contrast-hier-stops}), I've compressed or elided many of the featural labels due to the number of phonemes.
The coronals (on the far right) have all the labels explicitly given; this pattern is replicated in the labials, velars, and palatals.
Some possible phonemes (*/p',~ɠ/) have been given as ``◌'' (empty): these do not appear in native words, but are never allophones of other phonemes.
Presumably these unattested phonemes could appear in loanwords\footnote{but where is Utsišcy going to borrow */p', ɠ/ from?}.
One exception to the pattern occurs in the velars.
Although a voicing difference is to be expected between /k/ and */ɡ/, it does not occur.
I can't postulate an empty space here because /k/ is commonly realized as [ɡ].

Note that my analysis of the implosives (/ɓ,~ɗ,~ɟ/) is that they are [-continuant, +nonplosive] or, using more familiar names, [+stop, -plosive].
Although this might seem to be a contradictory analysis, I'm following Clements' paper ``Explosives, Implosives, and Nonexplosives: the Linguistic Function of Air Pressure Differences in Stops''.
There he analyzes implosives as [+stop, -obstruent], but given his definition of obstruence (positive air pressure) is slightly unfamiliar, I've opted to use a new term [plosive].
This is in keeping with the more accessible prose (esp. the title and abstract, which use ``non-explosive'' repeatedly), though I've stripped off the ``ex-'' prefix.
Unfortunately, my convention now distinguishes ``stop'' and ``plosive'', even though they are commonly synonymous; be aware.

\begin{figure}[H]
\centering
\Tree [.{+contoid, -sonorant, +continuant}
[.{\footnotesize +labial}
		[.{\scriptsize +gc} ◌ ] !\qsetw{0.5cm}
		[
			[.{\scriptsize +voice} v ] !\qsetw{0.7cm}
			[ f ] !\qsetw{0.5cm}
		]
	]
	[
		[.{\footnotesize +dorsal}
			[.{\footnotesize -dorsal}
				[.{\scriptsize +gc} x' ] !\qsetw{0.5cm}
				[
					[.{\scriptsize +voice} ʀ ] !\qsetw{0.7cm}
					[ x ] !\qsetw{0.5cm}
				]
			]
			[.{\footnotesize -velar}
				[.{\scriptsize +gc} ʃ' ] !\qsetw{0.5cm}
				[
					[.{\scriptsize +voice} ʒ ] !\qsetw{0.7cm}
					[ ʃ ] !\qsetw{0.5cm}
				]
			]
		]
		[.{\footnotesize -dorsal}
			[.{\footnotesize +gc} s' ]
			[.{\footnotesize -gc}
				[.{\footnotesize +voice} z ]
				[.{\footnotesize -voice} s ]
			]
		]
	]
]
\caption{Contrastive Hierarchy: fricatives}\label{fig:contrast-hier-continuants}
\end{figure}

The fricatives (fig.\ \ref{fig:contrast-hier-continuants}) have nearly the same structure as the stops.
The points of articulation are the same, and the only difference in the phonation is that fricatives lack a [plosive] feature.
Again, there are too many phonemes to be fully explicit with feature labels, so the coronals are the only fricatives fully labelled.
There is one more phonemic gap (*/f'/) which behaves the same as the gaps in the stops.

I've ordered place of articulation features (distinctive only in the obstruents) higher in the hierarchy than phonation features because phonological rules are more likely to assimilate phonation rather than place.


\begin{figure}[H]
\centering
\Tree [.{+contoid, +sonorant}
			[.{\footnotesize +nasal}
				[.{\footnotesize +labial} m ]
				[.{\footnotesize -labial} n ]
			] !\qsetw{2cm}
			[.{\footnotesize -nasal}
				[.{\footnotesize +glide}
					[.{\scriptsize +unvoice} j̊ ]
					[.{\scriptsize -unvoice} j ]
				] !\qsetw{1cm}
				[.{\footnotesize -glide}
					[.{\footnotesize +lateral}
						[.{\footnotesize +palatal}
							[.{\scriptsize +unvoice} ʎ̥ ]
							[.{\scriptsize -unvoice} ʎ ]
						]
						[.{\footnotesize -palatal}
							[.{\scriptsize +unvoice} l̥ ]
							[.{\scriptsize -unvoice} l ]
						]
					]
					[.{\footnotesize -lateral}
						[.{\scriptsize +unvoice} r̥ ]
						[.{\scriptsize -unvoice} r ]
					]
				]
			]
		]
\caption{Contrastive Hierarchy: sonorant consonants}\label{fig:contrast-hier-sonorants}
\end{figure}

In my quest to only use ``+'' for marked features and ``-'' for the corresponding unmarked, I've had to make some compromises with familiarity.
Whereas it is usual to see [±voice] but sonorants are naturally voiced, I've resorted to using [±unvoice] to distinguish /j̊,~l̥,~ʎ̥,~r̥/ from /j,~l,~ʎ,~r/ in the sonorants (fig.\ \ref{fig:contrast-hier-sonorants}).
Nevertheless, [voice] and [unvoice] are simple antonyms; phonological processes will not distinguish them.
The hierarchy for sonorants is otherwise straightforward.


\subsubsection{TODO Feature Summary}

TODO: come up with a list of all the features, then determine if, when a phone is given an additional feature, the phonetics can be altered.
Further, obtain the ratio of marked to unmarked features for each phoneme.

-sonsonant $\land$ +sonorant $\Leftrightarrow$ +syllabic

TODO:
About affricates: I'm not sure if they exist or not, but the could, since frication is not yet contrastive anywhere.
The only question would be how to analyze heterorganic stop-strictive clusters.

\begin{quote}
\subsubsection{Featural Notation:}

It might be useful to clarify the notational guides I'm using w.r.t. features.
Numerous forms are in use, and I can't be bothered to choose between them or understand the finer points of their theoretical assumptions.
I've come up with perhaps the worst coping strategy---invent my own convention with unfounded and unanalyzed assumptions---and here it is.

First, a feature might be the featural category: a simple enum (a finite set of unanalyzed tokens).
A feature might also be a featural value: a selection from a featural category.

Second, I'm going to spell categories capitalized but values lowercase.

Third, specifying a value from a catefory is written ``Category=value''.
When the category can be inferred from the value (e.g. ``coronal'' only appears in ``Place''), then the category can be omitted, but not the equals sign (e.g. ``=coronal'' instead of ``Place=coronal'').

Fourth, many categories thraditinoally (and in many theoretical frameworks, all categories) only have two values ``+,~-''.
By the rules so far, I would write these as e.g. ``High=+'', which is cumbersome.
Instead, (following the example) I select ``Height'' as the category, and ``+, -'' are synonyms for marked and unmarked respectively.
Then, to disambiguage the ``+, -'' values of various categories, suffix the lowercase traditional name of the marked form.
Finally, omit the equals sign.
Thus, instead of writing ``Height=+'', one writes ``+high'' if high is the marked height or ``-low'' if lowness is marked.

Note that mixing the sign convention (e.g. if highness is marked, using ``±low'') is dispreferred, as a sudden inconsistency might surprise a reader, leading them to think meta-textually.
If on the other hand, the feature has been pointed out as having odd markedness properties, then mixing is not so bad.
Sometimes a supposedly binary category has neutral elements markable with ``ø''; this symbol operates similarly to ``+, -''.
However, take care to specify if this value is transparent or opaque w.r.t. phonological processes operating on the ± values.
If there are both transparent and opaque values in a category, you'll have to make your own symbols.
\end{quote}



\section{Orthography}

Utsišcy's orthographies are mostly one-to-one phonemic, therefore I've decided to explain them here.
Utsišcy has both a romanization and a cyrillization.
Further, there is an ``asciization''---using only the glyphs of the ASCII character set---for use when diacritics are cumbersome or impossible to input to a computer system.

The non-phonemic exceptions in orthography capture regular phonetic processes in a way that is consistent with the script in use.
The phonemes /l̥, ʎ̥/, normally spelled «lh, llh» drop the <<-h>> when the phonetic context would imply they mjust be unvoiced.

There are a few special aesthetic considerations also.
In particular, /tʃ, cʃ/ clusters normally present as [tʃ], and so share the spelling «cš».
Additionally, the cyrillization leverages existing characters «ц, ч» for the clusters /ts/, [tʃ] since doing otherwise would read as highly unusual to anyone already familiar with cyrillic script.
The romanization does not use «q» or «qu» for /kʷ/ however.


TODO: if /j/ is going to attach to contoids, say as in /njɛt/, I'm not sure I want it spelled «nget», though that might be quirky/shibbolethy enough to be fun.
Further, using «w» just seems heavyweight to me. How could I get away writing /ʃʷatʃ/ not as «swacš», but as «suacš», «svacš», or something without compromising consonant clusters or not-diphthongs?
«ng̊et» and «sv̊acš»?

TODO when do I use capitalization?

\begin{table}[H]
\centering
\begin{subtable}[t]{.5\linewidth}
	\centering
	\begin{tabular}{ccccc}
		\hskip 0.7em m & \hskip 0.8em n \\
		p b & t d & c j & k \\
		\hskip 0.8em b' & t' d' & c' j' & k' \\
		f v & s z & š ž & x \\
		& \hskip 0.8em s' & \hskip 0.8em š' & x' \\
		& ř r & & & x̌ \\
		& lh l & ļ ļh \\
		w && ç g \\
	\end{tabular}
	\caption{consonants}\label{t:roman-consonants}
\end{subtable}
\hskip -6em
\begin{subtable}[t]{.5\linewidth}
	\centering
	\begin{tabular}{ccc}
		i & y u \\
		e & a o \\
	\end{tabular}
	\caption{vowels}\label{t:roman-vowels}
\end{subtable}

\begin{subtable}[t]{.5\linewidth}
	\centering
	\begin{tabular}{rc @{\hskip 3em} rc}
		tʃ & cš					&	tʃ' & cš' \\
		dʒ & ǰ \\
		C[-voice]l̥ & Cl̥		&	C[-voice]ʎ̥ & Cll \\
	\end{tabular}
	\caption{clusters (TODO)}\label{t:roman-clusters}
\end{subtable}
\caption{Romanization}\label{t:romanization}
\end{table}


\begin{table}[H]
\centering
\begin{subtable}[t]{.5\linewidth}
	\centering
	\begin{tabular}{ccccc}
		\hskip 0.7em m & \hskip 0.8em n \\
		p b & t d & c j & k \\
		\hskip 0.8em b' & t' d' & c' j' & k' \\
		f v & s z & sh zh & x \\
		& \hskip 0.8em s' & \hskip 1.2em sh' & x' \\
		& rh r & & & xh \\
		& lh l & llh ll \\
		w && ch g \\
	\end{tabular}
	\caption{consonants}\label{t:ascii-consonants}
\end{subtable}
\hskip -6em
\begin{subtable}[t]{.5\linewidth}
	\centering
	\begin{tabular}{ccc}
		i & y u \\
		e & a o \\
	\end{tabular}
	\caption{vowels}\label{t:ascii-vowels}
\end{subtable}

TODO notes on the odd character choices (j, g, ç, x̌, ř).
TODO comment on cedilla vs. comma below in unicode.


\begin{subtable}[t]{.5\linewidth}
	\centering
	\begin{tabular}{rc @{\hskip 3em} rc}
		tʃ & csh		&	tʃ' & csh' \\
	\end{tabular}
	\caption{clusters (TODO)}\label{t:ascii-clusters}
\end{subtable}
\caption{Asciization}\label{t:asciization}
\end{table}

\begin{table}[H]
\centering
\begin{subtable}[t]{.5\linewidth}
	\centering
	\begin{tabular}{ccccc}
		\hskip 0.7em м & \hskip 0.8em н \\
		п б & т д & ть дь & к \\
		\hskip 0.8em бӏ & тӏ дӏ & тьӏ дьӏ & кӏ \\
		ф в & с з & ш ж & х \\
		& сӏ \hskip 0.8em & шӏ \hskip 0.8em & хӏ \\
		& рхӏ р & & & гӏ \\
		& лхӏ л & лхӏь ль \\
		оъ && йхӏ й \\
	\end{tabular}
	\caption{consonants}\label{t:cyrillic-consonants}
\end{subtable}
\hskip -6em
\begin{subtable}[t]{.5\linewidth}
	\centering
	\begin{tabular}{ccc}
		и & і у \\
		е & а о \\
	\end{tabular}
	\caption{vowels}\label{t:cyrillic-vowels}
\end{subtable}

\begin{subtable}[t]{.5\linewidth}
	\centering
	\begin{tabular}{rc @{\hskip 3em} rc}
		ts & ц		&	ts' & цъ \\
		tʃ & ч		&	tʃ' & чъ \\
	\end{tabular}
	\caption{clusters (TODO)}\label{t:cyrillic-clusters}
\end{subtable}
\caption{Cyrillization (FIXME: tentative)}\label{t:cyrillization}
\end{table}

In the Cyrillization, probably replace yer (ъ) with palochka (ӏ), dependent on fonts (it's look a lot better than дьъ...).
I'm not sure about using «г» for /ʀ/.
Perhaps љ in place of ль.
I want ҋ in place of йхӏ: it's so much easier to read.
I know some language uses рхӏ exactly as I am here, but transporting that convension to the other liquids, though consistent, makes for some ugly n-graphs.
It's a bit extreme to use a digraph оъ for a labialization phoneme.

уциштьі

\section{Phonetics}

$$
\begin{drcases}
\mathipa{◌ʷʲ} \\
\mathipa{◌ʲʷ} \\
\end{drcases}
\to
\begin{cases}
\mathipa{[ɥ̊]} & \left/ {\left( \mathipa{C} \over \mathipa{-voice} \right)} \_\right. \\
\mathipa{[ɥ]} & \left/ {\left( \mathipa{C} \over \mathipa{+voice} \right)} \_ \right.  \\
\end{cases}
$$

$$
\begin{drcases}
\mathipa{◌ʷʲ} \\
\mathipa{◌ʲʷ} \\
\end{drcases}
\to
\left[{ \mathipa{ɥ} \over \alpha\mathipa{voice} }\right]
	\left/ {\left( \mathipa{C} \over \alpha\mathipa{voice} \right)} \_ \right.  \\
$$

The only thing, looking at these two options for typesetting rules is that $[{phone \over feature}]$ looks strange in the second.
My perspective is that features are phonemic rather than phonetic elements, and so should not be included in phonetic [...] transcription.



\end{document}
